%---------- Inleiding ---------------------------------------------------------

\section{Introductie} % The \section*{} command stops section numbering
\label{sec:introductie}

Locatie tracking is een populaire feature in de smartphone wereld. Vele apps maken er gebruik van, maar de meeste smartphones hebben ook automatische tracking services. Enkel een klein percentage van smartphone gebruikers weten dat dit bestaat. Deze feature staat standaard aan, en het is niet evident om te vinden waar men dit kan uitzetten. De meeste mensen hebben helemaal geen idee hoe veel bepaalde bedrijven weten over hun locatie (en andere gegevens).

Bij dit onderzoek zijn er volgende deelvragen om een gestructureerde conclusie te maken:

\begin{enumerate}
  \item Welke systemen en methodes worden gebruikt voor (mobiele) locatiegebaseerde diensten, en hoe werken deze?
  \item Hoe worden de verkregen gegevens opgeslaan, en worden deze (door derden) gebruikt?
  \item Wat mogen bedrijven doen met onze locatie gegevens? Houden de meest gekende apps zich hier aan?
  \item Waarom of wat is het nut van de automatische locatie-tracking op smartphones? Wat zegt de wet hierover?
\end{enumerate}

Na deze vragen zal er getest worden of het wel mogelijk is om de locatiegebaseerde diensten uit te zetten. Dit zal dan uitgelegd worden (adhv een website of artikel) zodat de meeste smartphone gebruikers dit kunnen toepassen. 

%---------- Stand van zaken ---------------------------------------------------

\section{State-of-the-art}
\label{sec:state-of-the-art}

In het begin zal verdiept worden in de locatie-tracking wereld. Er zijn reeds veel gelijkaardige studies die meer inzicht kunnen geven bij dit onderzoek. Bijvoorbeeld het artikel van ~\textcite{Yan2015}, dat uitleg geeft over hoe een locatie kan bepaald worden op basis van het totale stroomverbruik. Zo zijn er nog veel verschillende artikels die meer inzicht geven in de verschillende technieken. Google patents zullen zeer handig zijn om inzicht te krijgen in de structuur en methodiek van locatiegebaseerde diensten. Het patent van ~\textcite{Alan2005} geeft al een eerste grondige inleiding, onderaan staan veel gerelateerde patenten en artikels die zeker van pas zullen komen. Deze methodes zullen vooral gebruikt worden om na te gaan hoe onze locatie gegevens verkregen worden, en wat er daarna mee gebeurd. Ook zal gekeken worden wat de meest bekende apps/bedrijven (Google, Facebook, Swarm (Foursquare), ...) met onze locaties doen.
Daarna zal gezocht worden wat de wet zegt over deze locatiegebaseerde diensten.

%---------- Methodologie ------------------------------------------------------
\section{Methodologie}
\label{sec:methodologie}

Om een goed resultaat te krijgen wordt dus grotendeels gefocused op de systemen en methodes die gebruikt worden. Eerst zal er gekeken worden hoe zij tewerk gaan, welke service/sensoren er gebruikt worden. Wordt er gebruik gemaakt van machine learning? Hoe wordt alles opgeslaan?...

Wat bepaalde bedrijven met onze locatie gegevens doen zal opgezocht worden. Later zal gekozen worden welke specifieke bedrijven onderzocht worden, Facebook en Google zullen zeker besproken worden.
Er zal ook uitgelegd worden hoe een smartphone gebruiker bepaalde locatiegebaseerde diensten kan uitzetten.
Als laatste zal gekeken worden of het mogelijk is om als gewone gebruiker te zien wat een bepaald bedrijf bijhoudt van locatie gegevens over hen. Indien mogelijk zal beschreven worden hoe men deze gegevens kan verwijderen.



%---------- Verwachte resultaten ----------------------------------------------
\section{Verwachte resultaten}
\label{sec:verwachte_resultaten}
De systemen en methodes zullen zo goed mogelijk (mbv schemas) weergegeven worden.
Wat bepaalde bedrijven met onze locatie gegevens doen zal gestructureerd weergegeven worden, zodat het duidelijk is wie bepaalde data heeft en hoe dit wordt geanalyseerd.


%---------- Verwachte conclusies ----------------------------------------------
\section{Verwachte conclusies}
\label{sec:verwachte_conclusies}
Het doel van dit onderzoek is om na te gaan wat er met onze locatie gegevens gebeurd. 
Er wordt verwacht dat alles wat met deze gegevens gebeurd wettelijk toegestaan is.
Indien we de meeste grote bedrijven mogen geloven zal het mogelijk zijn om alles ivm locatie bepaling uit te zetten.


