%---------- Inleiding ---------------------------------------------------------

\section{Introductie} % The \section*{} command stops section numbering
\label{sec:introductie}

Locatie tracking is een populaire feature in de smartphone wereld. Vele apps maken er gebruik van, maar de meeste smartphones hebben ook automatische tracking services. Maar een klein percentage van de mensen weten dat hun smartphone dit heeft. Deze feature staat standaard aan, en het is niet evident om te vinden waar men dit kan uitzetten. De meeste mensen hebben geen idee hoe veel bepaalde bedrijven weten over hun locatie.

Bij dit onderzoek zijn er volgende deelvragen om een gestructureerde conclusie te maken:

\begin{enumerate}
  \item Welke systemen en methodes worden gebruikt voor mobile location-tracking, en hoe werken deze?
  \item Hoe worden de verkregen gegevens opgeslaan, en worden deze (door derden) gebruikt?
  \item Wat mogen bedrijven doen met onze locatie gegevens? En houden de meest gekende apps zich hier aan?
  \item Waarom of wat is het nut van de automatische locatie-tracking op smartphones? Wat zegt de wet hierover?
\end{enumerate}

Na deze vragen zal er getest worden of het volledig mogelijk is om de locatie bepaling services uit te zetten. 

%---------- Stand van zaken ---------------------------------------------------

\section{State-of-the-art}
\label{sec:state-of-the-art}

In het begin zal verdiept worden in de locatie-tracking wereld. Er zal gezocht worden naar gelijkaardige studies die meer inzicht kunnen geven bij dit onderzoek. Daarna worden de systemen en methodes opgezocht, dit voor zowel IOS als Android. Deze worden eventueel vergeleken, maar ze zullen vooral gebruikt worden om na te gaan wat er met onze locatie gegevens gebeurd. Ook zal gekeken worden wat de meest bekende apps/bedrijven (Google, Facebook, Swarm (Foursquare), ...) met onze locaties doen.
Daarna zal gezocht worden wat de wet zegt over deze location based services (LBS), alsook of dit in de gebruiks overeenkomst vermeld wordt.
Als laatste zal er getest worden of het mogelijk is om alle locatie bepaling services uit te zetten.

%---------- Methodologie ------------------------------------------------------
\section{Methodologie}
\label{sec:methodologie}

Om een goed resultaat te krijgen wordt dus grotendeels gefocused op de systemen en methodes die gebruikt worden. Eerst zal er gekeken worden hoe zij tewerk gaan, welke service/sensoren er gebruikt worden. Wordt er gebruik gemaakt van machine learning? Hoe wordt alles opgeslaan?...
De test, om te bepalen dat het mogelijk is om alle locatie bepaling services uit te zetten, zal gebeuren adhv een emulator en/of virtuele machine, waarin de locatie van de emulator of laptop kan veranderd worden. Alle locatie bepaling services zullen uitgezet worden, daarna zal uitgebreid onderzocht worden of nergens de huidige locatie is achterhaald.


%---------- Verwachte resultaten ----------------------------------------------
\section{Verwachte resultaten}
\label{sec:verwachte_resultaten}
De systemen en methodes zullen zo goed mogelijk (mbv schemas) weergegeven worden.
De resultaten van wat de bedrijven met onze gegevens doen zal in een tabel gegoten worden.
De test resultaten zullen zo volledig mogelijk beschreven worden, eerst hoe een bepaalde situatie werd opgesteld, en daarna wat er allemaal gebeurd is om na te gaan of de huidige locatie niet gelekt is.


%---------- Verwachte conclusies ----------------------------------------------
\section{Verwachte conclusies}
\label{sec:verwachte_conclusies}
Het doel van dit onderzoek is om na te gaan wat er met onze locatie gegevens gebeurd. 
Er wordt verwacht dat alles wat met deze gegevens gebeurd wettelijk toegestaan is.
Indien we de meeste grote bedrijven mogen geloven zal het mogelijk zijn om alles ivm locatie bepaling uit te zetten.
Er zal dus geen manier gevonden worden waardoor de locatie toch achterhaald werd.


