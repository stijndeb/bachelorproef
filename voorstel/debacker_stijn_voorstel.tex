%==============================================================================
% Sjabloon onderzoeksvoorstel bachelorproef
%==============================================================================
% Gebaseerd op LaTeX-sjabloon ‘Stylish Article’ (zie voorstel.cls)
% Auteur: Jens Buysse, Bert Van Vreckem

\documentclass[fleqn,10pt]{voorstel}

%------------------------------------------------------------------------------
% Metadata over het voorstel
%------------------------------------------------------------------------------

\JournalInfo{HoGent Bedrijf en Organisatie}
\Archive{2017 - 2018} % Of: Onderzoekstechnieken

%---------- Titel & auteur ----------------------------------------------------

% TODO: geef werktitel van je eigen voorstel op
\PaperTitle{Hoe goed kent een smartphone uw locaties?}
\PaperType{Onderzoeksvoorstel Bachelorproef} % Type document

% TODO: vul je eigen naam in als auteur, geef ook je emailadres mee!
\Authors{De Backer Stijn\textsuperscript{1}} % Authors
\CoPromotor{N.A.\textsuperscript{2} (N.A.)}
\affiliation{\textbf{Contact:}
  \textsuperscript{1} \href{mailto:stijn.debacker.w2871@student.hogent.be}{stijn.debacker.w2871@student.hogent.be};
  \textsuperscript{2} \href{N.A.}{N.A.};
}

%---------- Abstract ----------------------------------------------------------

\Abstract{
Vandaag de dag heeft bijna iedereen een smartphone, maar weten al deze gebruikers ook wat de smartphone allemaal over hen weet? In dit onderzoek wordt opzoek gegaan naar wat onze smartphone allemaal weet over onze locaties, hoe dit wordt gedaan en waarom. Wanneer weet de gsm om een bepaalde locatie op te slaan? Worden deze locaties enkel op de smartphone opgeslaan, of ook op de cloud? Is deze informatie privé of worden deze geanalyseerd door derden? De meeste smartphones hebben momenteel een sterke set van sensoren (WIFI, GPS, versnellings sensor, orientatie sensor,...), dan is de vraag natuurlijk, hoe werkt dit allemaal samen? Om hierop een antwoord te kunnen verwoorden zal verdiept worden in de wereld van locatie-tracking. Dit voor zowel IOS als Android.
Tijdens het onderzoek worden volgende deelvargen gesteld:
\begin{enumerate}

  \item Welke systemen en methodes worden gebruikt voor mobile location-tracking, en hoe werken deze?
  \item Hoe worden de verkregen gegevens opgeslaan, en worden deze (door derden) gebruikt?
  \item Wat mogen bedrijven doen met onze locatie gegevens? En houden de meest gekende apps zich hier aan?
  \item Waarom of wat is het nut van de automatische locatie-tracking op smartphones? Wat zegt de wet hierover?
  
\end{enumerate}
In de toekomst kan de uitkomst van deze deelvragen anders liggen, hoofdzakelijk omdat locatie tracking continue word bijgewerkt en ge-optimaliseerd.
Als deze vragen beantwoord zijn, zal er getest worden of het wel mogelijk is om alles van locatie-bepaling op een smartphone of laptop uit te zetten (zoals beweerd wordt door de meeste bedrijven).

}

%---------- Onderzoeksdomein en sleutelwoorden --------------------------------
% TODO: Sleutelwoorden:
%
% Het eerste sleutelwoord beschrijft het onderzoeksdomein. Je kan kiezen uit
% deze lijst:
%
% - Mobiele applicatieontwikkeling
% - Webapplicatieontwikkeling
% - Applicatieontwikkeling (andere)
% - Systeembeheer
% - Netwerkbeheer
% - Mainframe
% - E-business
% - Databanken en big data
% - Machineleertechnieken en kunstmatige intelligentie
% - Andere (specifieer)
%
% De andere sleutelwoorden zijn vrij te kiezen

\Keywords{Mobiele locatie tracking. Locaties --- IOS --- Android} % Keywords
\newcommand{\keywordname}{Sleutelwoorden} % Defines the keywords heading name

%---------- Titel, inhoud -----------------------------------------------------

\begin{document}

\flushbottom % Makes all text pages the same height
\maketitle % Print the title and abstract box
\tableofcontents % Print the contents section
\thispagestyle{empty} % Removes page numbering from the first page

%------------------------------------------------------------------------------
% Hoofdtekst
%------------------------------------------------------------------------------

% De hoofdtekst van het voorstel zit in een apart bestand, zodat het makkelijk
% kan opgenomen worden in de bijlagen van de bachelorproef zelf.
%---------- Inleiding ---------------------------------------------------------

\section{Introductie} % The \section*{} command stops section numbering
\label{sec:introductie}

Locatie tracking is een populaire feature in de smartphone wereld. Vele apps maken er gebruik van, maar de meeste smartphones hebben ook automatische tracking services. Maar een klein percentage van de mensen weten dat hun smartphone dit heeft. Deze feature staat standaard aan, en het is niet evident om te vinden waar men dit kan uitzetten. De meeste mensen hebben geen idee hoe veel bepaalde bedrijven weten over hun locatie.

Bij dit onderzoek zijn er volgende deelvragen om een gestructureerde conclusie te maken:

\begin{enumerate}
  \item Welke systemen en methodes worden gebruikt voor mobile location-tracking, en hoe werken deze?
  \item Hoe worden de verkregen gegevens opgeslaan, en worden deze (door derden) gebruikt?
  \item Wat mogen bedrijven doen met onze locatie gegevens? En houden de meest gekende apps zich hier aan?
  \item Waarom of wat is het nut van de automatische locatie-tracking op smartphones? Wat zegt de wet hierover?
\end{enumerate}

Na deze vragen zal er getest worden of het volledig mogelijk is om de locatie bepaling services uit te zetten. 

%---------- Stand van zaken ---------------------------------------------------

\section{State-of-the-art}
\label{sec:state-of-the-art}

In het begin zal verdiept worden in de locatie-tracking wereld. Er zal gezocht worden naar gelijkaardige studies die meer inzicht kunnen geven bij dit onderzoek. Daarna worden de systemen en methodes opgezocht, dit voor zowel IOS als Android. Deze worden eventueel vergeleken, maar ze zullen vooral gebruikt worden om na te gaan wat er met onze locatie gegevens gebeurd. Ook zal gekeken worden wat de meest bekende apps/bedrijven (Google, Facebook, Swarm (Foursquare), ...) met onze locaties doen.
Daarna zal gezocht worden wat de wet zegt over deze location based services (LBS), alsook of dit in de gebruiks overeenkomst vermeld wordt.
Als laatste zal er getest worden of het mogelijk is om alle locatie bepaling services uit te zetten.

%---------- Methodologie ------------------------------------------------------
\section{Methodologie}
\label{sec:methodologie}

Om een goed resultaat te krijgen wordt dus grotendeels gefocused op de systemen en methodes die gebruikt worden. Eerst zal er gekeken worden hoe zij tewerk gaan, welke service/sensoren er gebruikt worden. Wordt er gebruik gemaakt van machine learning? Hoe wordt alles opgeslaan?...
De test, om te bepalen dat het mogelijk is om alle locatie bepaling services uit te zetten, zal gebeuren adhv een emulator en/of virtuele machine, waarin de locatie van de emulator of laptop kan veranderd worden. Alle locatie bepaling services zullen uitgezet worden, daarna zal uitgebreid onderzocht worden of nergens de huidige locatie is achterhaald.


%---------- Verwachte resultaten ----------------------------------------------
\section{Verwachte resultaten}
\label{sec:verwachte_resultaten}
De systemen en methodes zullen zo goed mogelijk (mbv schemas) weergegeven worden.
De resultaten van wat de bedrijven met onze gegevens doen zal in een tabel gegoten worden.
De test resultaten zullen zo volledig mogelijk beschreven worden, eerst hoe een bepaalde situatie werd opgesteld, en daarna wat er allemaal gebeurd is om na te gaan of de huidige locatie niet gelekt is.


%---------- Verwachte conclusies ----------------------------------------------
\section{Verwachte conclusies}
\label{sec:verwachte_conclusies}
Het doel van dit onderzoek is om na te gaan wat er met onze locatie gegevens gebeurd. 
Er wordt verwacht dat alles wat met deze gegevens gebeurd wettelijk toegestaan is.
Indien we de meeste grote bedrijven mogen geloven zal het mogelijk zijn om alles ivm locatie bepaling uit te zetten.
Er zal dus geen manier gevonden worden waardoor de locatie toch achterhaald werd.




%------------------------------------------------------------------------------
% Referentielijst
%------------------------------------------------------------------------------
% TODO: de gerefereerde werken moeten in BibTeX-bestand ``voorstel.bib''
% voorkomen. Gebruik JabRef om je bibliografie bij te houden en vergeet niet
% om compatibiliteit met Biber/BibLaTeX aan te zetten (File > Switch to
% BibLaTeX mode)

\phantomsection
\printbibliography[heading=bibintoc]

\end{document}
