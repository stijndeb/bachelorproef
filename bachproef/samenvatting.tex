%%=============================================================================
%% Samenvatting
%%=============================================================================

% TODO: De "abstract" of samenvatting is een kernachtige (~ 1 blz. voor een
% thesis) synthese van het document.
%
% Deze aspecten moeten zeker aan bod komen:
% - Context: waarom is dit werk belangrijk?
% - Nood: waarom moest dit onderzocht worden?
% - Taak: wat heb je precies gedaan?
% - Object: wat staat in dit document geschreven?
% - Resultaat: wat was het resultaat?
% - Conclusie: wat is/zijn de belangrijkste conclusie(s)?
% - Perspectief: blijven er nog vragen open die in de toekomst nog kunnen
%    onderzocht worden? Wat is een mogelijk vervolg voor jouw onderzoek?
%
% LET OP! Een samenvatting is GEEN voorwoord!

%%---------- Nederlandse samenvatting -----------------------------------------
%
% TODO: Als je je bachelorproef in het Engels schrijft, moet je eerst een
% Nederlandse samenvatting invoegen. Haal daarvoor onderstaande code uit
% commentaar.
% Wie zijn bachelorproef in het Nederlands schrijft, kan dit negeren, de inhoud
% wordt niet in het document ingevoegd.

\IfLanguageName{english}{%
\selectlanguage{dutch}
\chapter*{Samenvatting}
\lipsum[1-4]
\selectlanguage{english}
}{}

%%---------- Samenvatting -----------------------------------------------------
% De samenvatting in de hoofdtaal van het document

\chapter*{\IfLanguageName{dutch}{Samenvatting}{Abstract}}

Voorlopige samenvatting:

Vandaag de dag heeft bijna iedereen een smartphone, maar weten al deze gebruikers ook wat de smartphone allemaal over hen weet? In dit onderzoek wordt gezocht wat onze smartphone allemaal weet over onze locaties, hoe dit wordt gedaan en waarom. Wanneer weet de gsm om een bepaalde locatie op te slaan? Worden deze locaties enkel op de smartphone opgeslaan, of ook op de cloud? Is deze informatie priv\'e of worden deze geanalyseerd door derden? De meeste smartphones hebben momenteel een sterke set van sensoren (WIFI, GPS, versnellings sensor, orientatie sensor,...), dan is de vraag natuurlijk, hoe werkt dit allemaal samen? Om hierop een antwoord te kunnen verwoorden zal verdiept worden in de wereld van locatiegebaseerde diensten.
Tijdens het onderzoek worden volgende deelvragen gesteld:
\begin{enumerate}

  \item Welke systemen en methodes worden gebruikt voor mobiele locatiegebaseerde diensten, en hoe werken deze?
  \item Hoe worden de verkregen gegevens opgeslaan, en worden deze (door derden) gebruikt?
  \item Wat mogen bedrijven doen met onze locatie gegevens? Houden de meest gekende apps zich hier aan?
  \item Waarom of wat is het nut van de automatische locatie-tracking op smartphones? Wat zegt de wet hierover?
  
\end{enumerate}
In de toekomst kan de uitkomst van deze deelvragen anders liggen, hoofdzakelijk omdat locatiegebaseerde diensten continue worden bijgewerkt en ge-optimaliseerd.
Als deze vragen beantwoord zijn, zal er getest worden of het wel mogelijk is om alles van locatie-bepaling op een smartphone of laptop uit te zetten.
